% Options for packages loaded elsewhere
\PassOptionsToPackage{unicode}{hyperref}
\PassOptionsToPackage{hyphens}{url}
%
\documentclass[
]{article}
\usepackage{amsmath,amssymb}
\usepackage{iftex}
\ifPDFTeX
  \usepackage[T1]{fontenc}
  \usepackage[utf8]{inputenc}
  \usepackage{textcomp} % provide euro and other symbols
\else % if luatex or xetex
  \usepackage{unicode-math} % this also loads fontspec
  \defaultfontfeatures{Scale=MatchLowercase}
  \defaultfontfeatures[\rmfamily]{Ligatures=TeX,Scale=1}
\fi
\usepackage{lmodern}
\ifPDFTeX\else
  % xetex/luatex font selection
\fi
% Use upquote if available, for straight quotes in verbatim environments
\IfFileExists{upquote.sty}{\usepackage{upquote}}{}
\IfFileExists{microtype.sty}{% use microtype if available
  \usepackage[]{microtype}
  \UseMicrotypeSet[protrusion]{basicmath} % disable protrusion for tt fonts
}{}
\makeatletter
\@ifundefined{KOMAClassName}{% if non-KOMA class
  \IfFileExists{parskip.sty}{%
    \usepackage{parskip}
  }{% else
    \setlength{\parindent}{0pt}
    \setlength{\parskip}{6pt plus 2pt minus 1pt}}
}{% if KOMA class
  \KOMAoptions{parskip=half}}
\makeatother
\usepackage{xcolor}
\usepackage[margin=1in]{geometry}
\usepackage{longtable,booktabs,array}
\usepackage{calc} % for calculating minipage widths
% Correct order of tables after \paragraph or \subparagraph
\usepackage{etoolbox}
\makeatletter
\patchcmd\longtable{\par}{\if@noskipsec\mbox{}\fi\par}{}{}
\makeatother
% Allow footnotes in longtable head/foot
\IfFileExists{footnotehyper.sty}{\usepackage{footnotehyper}}{\usepackage{footnote}}
\makesavenoteenv{longtable}
\usepackage{graphicx}
\makeatletter
\newsavebox\pandoc@box
\newcommand*\pandocbounded[1]{% scales image to fit in text height/width
  \sbox\pandoc@box{#1}%
  \Gscale@div\@tempa{\textheight}{\dimexpr\ht\pandoc@box+\dp\pandoc@box\relax}%
  \Gscale@div\@tempb{\linewidth}{\wd\pandoc@box}%
  \ifdim\@tempb\p@<\@tempa\p@\let\@tempa\@tempb\fi% select the smaller of both
  \ifdim\@tempa\p@<\p@\scalebox{\@tempa}{\usebox\pandoc@box}%
  \else\usebox{\pandoc@box}%
  \fi%
}
% Set default figure placement to htbp
\def\fps@figure{htbp}
\makeatother
\setlength{\emergencystretch}{3em} % prevent overfull lines
\providecommand{\tightlist}{%
  \setlength{\itemsep}{0pt}\setlength{\parskip}{0pt}}
\setcounter{secnumdepth}{-\maxdimen} % remove section numbering
\usepackage{bookmark}
\IfFileExists{xurl.sty}{\usepackage{xurl}}{} % add URL line breaks if available
\urlstyle{same}
\hypersetup{
  pdftitle={Sleep Duration, Difficulty, and Health Correlates Among U.S. Adults (NHANES 2009--2011)},
  hidelinks,
  pdfcreator={LaTeX via pandoc}}

\title{Sleep Duration, Difficulty, and Health Correlates Among U.S.
Adults (NHANES 2009--2011)}
\author{Yianni Papagiannopoulos\\
Madhavan Narkeeran\\
Alexandra Julka}
\date{October 15, 2025}

\begin{document}
\maketitle

{
\setcounter{tocdepth}{3}
\tableofcontents
}
\begin{center}\rule{0.5\linewidth}{0.5pt}\end{center}

\subsection{\texorpdfstring{\textbf{Introduction}}{Introduction}}\label{introduction}

Sleep is a crucial factor influencing physical, emotional, and cognitive
health. Adequate sleep supports immunological response, metabolic
balance, and mental resilience, while long-term sleep loss is associated
with obesity, heart disease, anxiety, and depression (Cappuccio et al.,
2010; Itani et al., 2017). Stressors, digital exposure, and changing
work habits have all been recently linked to the widespread sleep
disruptions that affect the length and quality of sleep in modern
societies.

This project uses data from the \texttt{NHANESraw} (2009--2011) survey
dataset, which is conducted by the Centers for Disease Control and
Prevention (CDC) to monitor health and nutrition trends in the United
States (Centers for Disease Control and Prevention {[}CDC{]}, 2013). The
dataset contains 20,293 responses (including all of the missing values).
The purpose of this analysis is to investigate and illustrate the
relationship between sleep patterns and demographic characteristics,
lifestyle choices, and overall health, not to establish cause and
effect. The assumption, however, is that BMI (Body-Mass-Index), mental
health, age, gender, and amount of physical activity all have a
significant effect on sleep.

\subsection{\texorpdfstring{\textbf{Background}}{Background}}\label{background}

The Centers for Disease Control and Prevention (CDC) operates the
National Center for Health Statistics (NCHS) program to administer the
National Health and Nutrition Examination Survey (NHANES). The U.S.
population's health and nutritional status are assessed through a
combination of laboratory testing, structured interviews, and physical
examinations (Centers for Disease Control and Prevention {[}CDC{]},
2013). The NHANES incorporates both self-reported and objective medical
data, in contrast to the majority of health surveys which only utilize
questionnaires. This gives a comprehensive picture of how biological and
lifestyle factors interact to affect the overall health of an
individual.

\subsection{\texorpdfstring{\textbf{Data and Cleaning
Strategy}}{Data and Cleaning Strategy}}\label{data-and-cleaning-strategy}

The NHANESraw dataset contained numerous missing values and data
inconsistencies. For the purpose of preserving as many observations as
possible, data cleaning was approached on a question-by-question basis.
Rows with missing values were eliminated only for the variables
pertinent to each analysis, as opposed to eliminating all incomplete
records at once. For numerical variables, extreme outliers were removed
to enhance the quality of the data. For instance, observations reporting
less than 2 hours or more than 15 hours of sleep were not included in
the \texttt{SleepHrsNight} variable because these values were deemed
unrealistic.

To make comparisons easier, some variables were also re-coded or
normalized. For example, in order to enable faceted visualizations by
age range, the Age variable was binned into an \texttt{AgeGroup} column
with categories of 16--24, 25--34, 35--44, 45--54, 55--64, 65--74, and
75+. Furthermore, inconsistent labels for categorical variables were
standardized. The Race1 variable, which had distinct entries for
``Mexican'' and ``Hispanic,'' was one such example. For the purposes of
maintaining consistency, all ``Mexican'' entries were recorded as
``Hispanic,'' since ``Mexican'' is a subgroup of ``Hispanic.''

Overall, these focused cleaning and normalization processes enhanced
dataset comparability and reliability across analyses, ensuring that the
outcomes represented real relationships rather than artifacts.

\subsection{\texorpdfstring{\textbf{Guiding Research
Questions}}{Guiding Research Questions}}\label{guiding-research-questions}

This project aims to explore how demographic, behavioral, and
health-related factors correlate with sleep duration and sleep
difficulty among U.S. adults. The analysis began with ten exemplary
questions, which were subsequently refined to seven primary research
questions. Specifically, this study addresses the following:

\begin{enumerate}
\def\labelenumi{\arabic{enumi}.}
\item
  \textbf{How does reported sleep duration relate to reported difficulty
  sleeping across age groups?}
\item
  \textbf{How specifically does Reported General Health associate with
  Reported Sleep Trouble?}
\item
  \textbf{How does the proportion of reported trouble sleeping vary by
  age group for males versus females?}
\item
  \textbf{Among adults aged 21 years or older, what patterns in sleep
  duration are observed across different lifestyle groups (smokers,
  drinkers, both, or neither)?}
\item
  \textbf{What is the relationship between reported physical activity
  and average sleep duration among adults in the United States?}
\item
  \textbf{What association, if any, exists between daily screen time and
  nightly sleep duration?}
\item
  \textbf{What is the observable relationship, if any, between reported
  average sleep duration and average blood pressure (systolic and
  diastolic) in adult participants?}
\end{enumerate}

Additional exploratory questions and visualizations that were
considered, but not included in the final analysis are presented in the
Appendix.

\subsection{\texorpdfstring{\textbf{Results and
Visualizations}}{Results and Visualizations}}\label{results-and-visualizations}

\textbf{QUESTION 1}

Out of the six guided questions, three of them were ones that matched
the initial assumptions for this project. The other four were somewhat
of a surprise. The first question, for instance, explored the
relationship between age groups and difficulty sleeping. The initial
assumption was that age had a significant effect on sleep and that
participants that reported having trouble sleeping slept for a fewer
number of hours on average. The visualization and statistical test
confirmed this assumption.

\begin{figure}
\includegraphics[width=1\linewidth]{DATA350_Project_1_files/figure-latex/fig1_sleep_age-1} \caption{<b>Figure 1. Average Sleep Duration by Age and Reported Difficulty Sleeping</b>}\label{fig:fig1_sleep_age}
\end{figure}

\emph{Mean nightly sleep duration (±95\% confidence intervals) by age
group and self-reported trouble sleeping. Respondents who reported
having trouble sleeping consistently averaged fewer hours of nightly
rest across all age groups, with both groups showing a slight rebound in
older adulthood (NHANES 2009--2011).}

As might be expected, participants who did not report having sleep
problems, typically slept longer per night than people who reported
having difficulty. Among those without reported sleep difficulty, older
adults (aged 65 years or older) disclose a larger average sleep duration
per night than the younger participants. However, while the younger
participants typically slept longer on average per night than older
adults, the relationship is reversed among those who have reported
having sleep issues.

This discrepancy may be the result of lifestyle and general health
factors. While older adults with sleep problems may experience comorbid
health conditions that further disrupt sleep, younger participants with
reported sleep trouble are typically less impacted by chronic physical
conditions caused by age, that can further disrupt rest. On the other
hand, compared to younger or middle-aged adults, older participants (65
years old or higher) without reported sleep disturbances may benefit
from having more off time due to retirement, for example. Participants
in their middle years (45-64 years old) received the least amount of
sleep per night, which may be the result of potential time constraints
and other related health conditions.

\textbf{QUESTION 2}

Similarly, the second question also demonstrated to be what was expected
initially. Assuming mental and physical health affects sleep, it was
deduced that there should exist a positive correlation between a
participant's reported general health and reported sleep trouble. The
visualization and analysis is shown below.

\begin{figure}
\includegraphics[width=1\linewidth]{DATA350_Project_1_files/figure-latex/fig8_sleep_health-1} \caption{<b>Figure 2. Reported Sleep Trouble by Self-Reported General Health (NHANES 2009–2011)</b>}\label{fig:fig8_sleep_health}
\end{figure}

\emph{Percent that reported Sleep Trouble for each category of
Self-Reported General Health. Error bars are 95\% confidence intervals
(NHANES 2009--2011).}

It can be observed that among people who have reported having a lower
general health have also reported having a higher percentage of reported
sleep trouble. To verify statistical significance, the following
statistical test was performed.

\textbf{Statistical Test}: Chi-Squared Test for Trend (Cochran--Armitage
trend test)

\textbf{Hypothesis:}

\textbf{H₀:} There is no linear trend in reported sleep trouble
percentage across reported general health categories.

\textbf{H₁}: There is a positive linear trend (reported sleep trouble
percentage increases as reported general health worsens).

Testing at significance level = 0.05.

\textbf{Conclusion:}

The p-value was found to be less than 2.2e-16 which is much less than
the significance level = 0.05. Therefore, there is very strong evidence
to conclude that the percentage of participants reported having trouble
sleeping increases as their reported general health worsens.

\textbf{QUESTION 3}

\begin{figure}
\includegraphics[width=1\linewidth]{DATA350_Project_1_files/figure-latex/ig3_sleep_gender-1} \caption{<b>Figure 3. Reported Trouble Sleeping by Age Group (16+) and Gender</b>}\label{fig:ig3_sleep_gender}
\end{figure}

\emph{Proportion of reported trouble sleeping across different age
groups for Males and Females. Error bars are 95\% confidence intervals
(NHANES 2009--2011).}

Based on Figure 3, females across all age groups exhibited a higher
proportion of reported trouble sleeping compared to males. To determine
whether these differences are statistically significant, a series of
two-proportion z-tests were conducted for each age group separately at a
significance level of 0.05. The null hypothesis states that females and
males have equal proportions of reported sleep trouble, while the
alternative hypothesis posits that females have a higher proportion of
reported sleep trouble than males. The results of these tests are
summarized in the table below:

\begin{longtable}[]{@{}ll@{}}
\toprule\noalign{}
\endhead
\bottomrule\noalign{}
\endlastfoot
\textbf{Age Group} & \textbf{P-Value} \\
16-24 & 0.000157 \\
25-34 & 0.000104 \\
35-44 & 0.00000141 \\
45-54 & 0.000288 \\
55-64 & 0.00000582 \\
65-74 & 0.0000101 \\
75+ & 0.00297 \\
\end{longtable}

The findings demonstrated that both proportional z-tests had p-values
that were significantly below the significance threshold level of 0.05.
Consequently, across all age groups, there is statistically significant
evidence to suggest that women reported a higher percentage of sleep
difficulties than men.

It is crucial to take into account that the dependent variable does not
reflect objectively measured sleep duration or quality, but rather
self-reported sleep issues. Therefore, even though the analysis shows
that women are more likely to report having trouble sleeping, this does
not necessarily mean that they actually have more sleep problems
compared to their male counterparts. These results might be the result
of disparities in perceptions, health awareness, or symptom disclosure
willingness.

A Nuffield Health article titled ``5 Reasons Men Avoid Going to the
Doctor,'' published in 2024, supports this interpretation by claiming
that approximately 65\% of men put off seeking medical attention when
they are ill, pointing to a larger trend of male under reporting.
Therefore, rather than being due to physiological differences, the
observed gender disparity in reported sleep trouble may be partially
caused by social or behavioral factors.

In terms of age, the pattern was more predictable: middle-aged
participants reported the highest rates of sleep problems, which is in
line with the previous assumptions that this particular age group sleeps
less on average than younger and older adults due to time constraints or
health conditions.

\textbf{QUESTION 4}

Results from Question 4 deviated from the original hypothesis.
Initially, it was presumed that participants who smoked or drank would
also have reported less regular or healthier sleep patterns compared to
those that did neither. The density plot and related statistical test,
however, indicate otherwise.

\begin{figure}
\includegraphics[width=1\linewidth]{DATA350_Project_1_files/figure-latex/fig9_sleep_lifestyle-1} \caption{<b>Figure 4. Sleep Duration Distribution by Lifestyle (Adults 21+)</b>}\label{fig:fig9_sleep_lifestyle}
\end{figure}

\emph{Kernel density plot displaying the distribution of nightly sleep
duration among adult participants aged 21 years or older, grouped by
lifestyle category (smoker only, drinker only, both, or neither). The
normalized probability density function is represented by each curve,
with vertical lines denoting the average amount of sleep for each group.
In contrast to non-smokers and non-drinkers, smokers and drinkers
exhibit wider and more irregular distributions, suggesting greater
variability in sleep patterns, even though mean values are similar
across lifestyles (NHANES 2009--2011).}

Compared to drinkers or people who do neither, the distribution for
smokers is noticeably smaller, suggesting that the majority of smokers
sleep roughly seven hours every night with less variation among the
group. On the other hand, there is more variation among those who
reported drinking, with noticeable peaks occurring at six, seven, and
eight hours of sleep on average per night. Compared to smokers, even
those who claimed did not smoke or drink had more varied sleep duration.

These results were unexpected since it had been assumed that poorer
sleep habits were associated with unhealthier lifestyles. This result
emphasizes a crucial point: irregularities or bad habits in one area of
life do not always indicate unhealthy behavior in another. Although this
is still purely conjectural, it is plausible that smokers in this
instance follow more regimented routines, which may be connected to the
habitual character of their behavior.\\

\textbf{QUESTION 5}

The results for the next questions were a bit surprising. For Question 5
in particular, it was expected that physical activity positively affects
sleep quantity and quality. However, the box plot below seems to
indicate that there is actually no correlation between them.

\begin{figure}
\centering
\pandocbounded{\includegraphics[keepaspectratio]{DATA350_Project_1_files/figure-latex/figure6-physactivity-vs-sleep-1.pdf}}
\caption{Figure 5. Reported Physical Activity vs.~Sleep Duration}
\end{figure}

\emph{Boxplot comparing average hours of sleep reported among adults who
claimed to be physically active versus those who did not. The mean sleep
durations (red lines) are nearly identical between groups, suggesting
that reported physical activity level does not substantially influence
total sleep time (NHANES 2009--2011).}

These results are notable because many studies indicate that more
exercise results in better sleep quality. However, there are some key
considerations to take part. Average sleep duration is a measure of
healthy sleep habits, but it is not the primary measure of a person's
health. After all, sleep quality also needs to be considered, which can
differ drastically from sleep quantity. Second, sleep is the dependent
variable in this case. It is possible that sleep has more of an effect
on physical activity, than physical activity has on sleep. Sleep is the
cause of a variety of different factors, physical prowess being one of
them. Therefore, a key takeaway may be that a lack of significant
association in one direction does not preclude a meaningful causal
relationship in the other in retrospect to average hours of sleep per
night.

\textbf{QUESTION 6}

The results to Question 6 was possibly the most surprising out of all
the results. The assumption was that screen usage would impact sleep.
This is because many electronic devices emit blue light, which
diminishes the amount of melatonin in the body. Melatonin, for
reference, helps regulate the sleeping cycle and aids the body in
falling asleep (Cappuccio et al.). However, the visualization below
tells a different story.

\begin{figure}
\includegraphics[width=1\linewidth]{DATA350_Project_1_files/figure-latex/fig7_screen_sleep-1} \caption{<b>Figure 6. Daily Screen Time (Computer + TV) vs. Sleep Duration</b>}\label{fig:fig7_screen_sleep}
\end{figure}

\emph{Heat map illustrating the relationship between total daily screen
time (hours spent watching TV and using a computer) and average nightly
sleep duration among participants aged 16 and older. The percentage of
respondents who fall into that screen time--sleep range is shown by each
tile. The majority of people cluster around 7 hours of sleep per night,
according to the fitted regression line (slope ≈ 0.005,} R² \emph{≈
0.004), which indicates no discernible relationship between total screen
exposure and sleep duration (NHANES 2009--2011)}

As depicted by the visualization, the regression line is relatively
flat, indicating that as reported screen time increases, average sleep
duration does not change substantially. Although this finding may come
as a surprise, it is consistent with the knowledge that sleep quantity
and quality are two distinct metrics. While total sleep hours may appear
unchanged, increased screen exposure, especially before bedtime, can
still impair sleep quality (e.g., restfulness or sleep latency). It's
also crucial to consider that the dataset used in this analysis was
taken from the 2009--2011 NHANES cycle, which only tracked screen time
on computers and televisions, which does not include smartphone use.
Since this time frame precedes the widespread adoption of most modern
smartphones, the relationship between screen exposure and sleep duration
may appear weaker than it would in more recent data.

\textbf{Question 7}

It was also expected that blood pressure would have a noticeable effect
on sleep. There are two values that measure someone's blood pressure:
systolic and diastolic blood pressure. According to an article by
VeryWellHealth on systolic vs diastolic blood pressure, systolic blood
pressure is more important to doctors because it measures the heart when
it's at its highest pressure point (during the heartbeat). However, for
this report, both systolic and diastolic pressures are analyzed. The
results are shown below.

\begin{figure}
\includegraphics[width=1\linewidth]{DATA350_Project_1_files/figure-latex/fig4_sleep_bp-1} \caption{<b>Figure 7. Sleep Hours vs. Blood Pressure (Adults 16+)</b>}\label{fig:fig4_sleep_bp}
\end{figure}

\emph{Scatterplots displaying the relationship between nightly sleep
duration (hours) and average blood pressure (systolic \& diastolic)
among adults aged 16 and older. An ordinary least squares (OLS)
regression line with blue bands representing 95\% confidence intervals
is included in each panel. While there are weak negative correlations
between sleep duration and both diastolic and systolic blood pressure (p
\textless{} 0.01), the data are not clearly linear and are widely
scattered. The regression lines only explain less than 1\% of the
variance and summarize a very slight downward trend, suggesting that
blood pressure is not a reliable indicator of sleep duration}.

\begin{longtable}[]{@{}
  >{\raggedright\arraybackslash}p{(\linewidth - 0\tabcolsep) * \real{1.0042}}@{}}
\toprule\noalign{}
\endhead
\bottomrule\noalign{}
\endlastfoot
\begin{minipage}[t]{\linewidth}\raggedright
\textbf{Statistical Test:} Pearson's correlation and simple linear
regression were performed for each blood pressure type.

\begin{itemize}
\item
  Diastolic (mmHg): r = −0.092, p \textless{} 0.001, n = 11,773, slope =
  −0.011 hours/mmHg (≈ −0.66 min/mmHg), R² = 0.008
\item
  Systolic (mmHg): r = −0.029, p = 0.002, n = 11,773, slope = −0.0023
  hours/mmHg (≈ −0.14 min/mmHg), R² = 0.001
\end{itemize}

Although both tests reached statistical significance due to the large
sample size, their effect sizes were negligible in practical terms. This
suggests that blood pressure explains less than 1\% of the variance in
reported sleep duration.
\end{minipage} \\
\end{longtable}

The expectation was that both measurements of blood pressure would have
a negative effect on how long participants slept per night (so, the
higher the blood pressure, the fewer the hours of sleep). However, as
can be seen in the scatterplots above, there is a negligible negative
correlation between both measurements and sleep. The p-value is
relatively low, which typically indicates there exists a significant
correlation between the two variables. However, the relatively low R²
value may be influenced by the dataset, which contains thousands of
observations after missing data is excluded. Considering these factors,
blood pressure appears to have no profound effect on sleep duration.
This finding was somewhat unexpected, however it should be noted that
individuals largely have control over their sleep schedules. Many people
are aware of the benefits associated with obtaining approximately eight
hours of sleep per night, and actively strive to meet this goal. Blood
pressure, by contrast, is unlikely to directly influence the decision of
when to go to bed. Instead, it may be more compelling to examine how
sleep may affect blood pressure, since blood pressure is not a factor
most people can directly control, unlike the decision of when to go to
bed.\\

\subsection{\texorpdfstring{\textbf{Limitations and Sources of
Error}}{Limitations and Sources of Error}}\label{limitations-and-sources-of-error}

When interpreting the findings of this analysis, it is important to take
into account a number of limitations and possible sources of error.

First, some variables in the \texttt{NHANESraw} dataset used in this
study are dated, as it was gathered between 2009 and 2011. Given the
significant changes in smartphone use and digital habits since then,
this is especially pertinent for metrics such as daily screen time. As a
result, some relationships, including those between screen time and
sleep duration, may not fully reflect patterns observed in 2025. Second,
as stated earlier, self-reported data from questionnaires and interviews
make up a portion of the dataset. Recall bias, social desirability bias,
and rounding errors may affect these kinds of data. For example,
participants may tend to estimate total hours of sleep instead of
reporting precise durations.

Furthermore, although precautions were taken to guarantee data
consistency, such as eliminating missing entries based on
question-specific criteria and filtering implausible values, these
actions might have resulted in selection bias or decreased the
representativeness of particular subgroups. Additionally, re-coding and
grouping variables (e.g., binning age into categories or combining
``Mexican'' into ``Hispanic'') made the data easier to analyze, but they
might have masked more subtle differences within groups. To add on to
this, the NHANES' intricate survey weights were not applied; instead,
the dataset was analyzed as a straightforward random sample. Therefore,
the results are not strictly generalizable to the entire U.S.
population; rather, they describe associations within the sample itself.

Next, cross-sectional data, which offers a moment in time, forms the
basis of the study. Although correlations can be found, such as between
blood pressure and sleep duration, causal relationships cannot be
deduced. Results may also be impacted by unmeasured confounding factors
such as stress, occupation, or long-term medical conditions. Finally,
human judgment in establishing thresholds, exclusions, and groupings
invariably introduces some subjectivity, even in the face of efforts to
maintain objectivity during data cleaning and interpretation.

All things considered, even though these restrictions limit the accuracy
and generalizability of some findings, the findings nevertheless provide
insightful information about the sleep patterns and their relationships
to various demographic and health characteristics among American adults.

\subsection{\texorpdfstring{\textbf{Conclusion}}{Conclusion}}\label{conclusion}

Despite growing public awareness, many aspects of how sleep interacts
with behavior and self-perceived health remain unclear. For instance,
people may have comparable sleep schedules but greatly differ in their
degrees of wellbeing and rest. The importance of looking at both hours
slept and self-reported difficulty sleeping is highlighted by the
distinction between sleep quantity and quality. This distinction is
especially important to remember because confusing sleep quality and
quantity can lead to dangerous assumptions. For example, one of the
results indicates that screen time and hours of sleep do not correlate.
If taken at face value, people may assume that sleep and screen time do
not correlate. However, this assumption is false and dangerous because
there are a multitude of studies that show how blue light negatively
affects melatonin production. Sleep quantity is easier to measure, but
that doesn't mean it should be the only value used to explain sleep.
After all, sleep is a mysterious phenomenon that to this day still
stumps the scientific community. Sleep quality is arguably just as, if
not more, important than sleep quantity. It cannot be ignored simply
because some researchers want to cut corners.

In addition, it's important to keep in mind that causal relationships
don't always go both ways. For instance, it's heavily implied in an
article written for the Cleveland Clinic that sleep has a significant
effect on blood pressure. However, the results of this project indicate
that blood pressure has a limited effect on the number of hours a person
sleeps. The main takeaway of this project is to never take common sense
for granted. It's always better to test theories, no matter how obvious
they may seem at first, because the results may be different than
expected. A good test may include certain statistical methods, but an
even great test is to simply visualize the theory by plotting a dataset.
As the famous saying goes, ``a picture is worth a thousand words''.

\subsection{\texorpdfstring{\textbf{Reproducibility and Code
Availability}}{Reproducibility and Code Availability}}\label{reproducibility-and-code-availability}

All the source code is available on
\href{https://github.com/jpapagia/DATA350_Project1.git}{GitHub} with the
main Rmd file to reproduce all the results from this report as well as
separate R files to be able to reproduce results for each individual
question explored.\\

This analysis was conducted in R utilizing the following packages:
\texttt{tidyverse}, \texttt{gridExtra}, \texttt{scales}, and
\texttt{broom}.\\

The included \texttt{NHANESraw.csv} dataset must be in the working
directory before execution in order for the results to be reproduced.
All of the figures, tables, and statistical outputs in this report can
be completely recreated with these dependencies and resources in place.

\subsection{\texorpdfstring{\textbf{Appendix}}{Appendix}}\label{appendix}

Although they were created during the analysis's exploratory stage, the
following analysis questions and accompanying visualizations were
eventually omitted from the final report. These figures are included
here to show the variety of exploratory methods used on the NHANES
dataset and to document the larger analytical process. Although they
offered helpful intermediate insights, they either failed to identify
strong patterns, were redundant with other visuals, or were not
sufficiently informative for the research questions.

\textbf{What is the relationship between average sleep duration and the
number of poor mental health days, and how does this relationship differ
across age groups?}\\

\begin{figure}
\includegraphics[width=1\linewidth]{DATA350_Project_1_files/figure-latex/fig2_sleep_mentalhealth-1} \caption{<b>Figure 8. Sleep Duration vs. Days of Poor Mental Health (Past 30 Days) per Age Group</b>}\label{fig:fig2_sleep_mentalhealth}
\end{figure}

\emph{Scatterplot with LOESS smoothing lines showing the association
between average nightly sleep duration and the number of self-reported
days of poor mental health within a 30 day period, separated by age
group. Across all groups, much shorter or longer sleep durations than
the average were generally associated with more days reported with poor
mental health. On average, the lowest number of poor mental health days
in the 30 day period occurred when participants revived nearly 7 hours
of sleep per night. The accompanying table lists Spearman correlation
coefficients (ρ), p-values, and sample sizes for each age group (NHANES
2009--2011).}

\textbf{How does the proportion of reported trouble sleeping vary by age
group for different races?}

\begin{figure}
\includegraphics[width=1\linewidth]{DATA350_Project_1_files/figure-latex/fig3_sleep_race-1} \caption{<b>Figure 9. Reported Trouble Sleeping by Age Group (16+) and Race/Ethnicity</b>}\label{fig:fig3_sleep_race}
\end{figure}

\emph{Bar chart showing the proportion of adults reporting trouble
sleeping by race/ethnicity across seven age groups. Mexican participants
were merged into the Hispanic category. Each bar represents the
estimated percentage within a group, with 95 \% Wald confidence
intervals shown as error bars. Prevalence of reported sleep trouble
tends to increase from young adulthood through middle age and remains
elevated into older adulthood, with White respondents generally
reporting the highest rates across most age categories (NHANES
2009--2011).}

\textbf{How does sleep duration vary with BMI categories across
different genders?}

\begin{figure}
\centering
\pandocbounded{\includegraphics[keepaspectratio]{DATA350_Project_1_files/figure-latex/figure5-bmi-vs-sleep-1.pdf}}
\caption{Figure 10. Sleep Duration vs Body Mass Index (BMI) by Gender}
\end{figure}

\emph{Average nightly sleep hours across BMI categories, separated by
gender. Each violin represents the distribution of self-reported sleep
duration within a BMI group, with the box showing the interquartile
range and the white dot marking the mean. Data from NHANES
(2009--2011).}

\subsection{\texorpdfstring{\textbf{References}}{References}}\label{references}

\begin{itemize}
\item
  Buysse, D. J. (2014). Sleep health: Can we define it? Does it matter?
  Sleep, 37(1), 9--17. \url{https://doi.org/10.5665/sleep.3298}
\item
  Cappuccio, F. P., D'Elia, L., Strazzullo, P., \& Miller, M. A. (2010).
  Sleep duration and all-cause mortality: A systematic review and
  meta-analysis of prospective studies. Sleep, 33(5), 585--592.
  \url{https://doi.org/10.1093/sleep/33.5.585}
\item
  Clinic, C. (2023, February 13). How a Lack of Sleep Contributes to
  High Blood Pressure. Cleveland Clinic; Cleveland Clinic.
  \url{https://health.clevelandclinic.org/can-lack-of-sleep-cause-high-blood-pressure}
\item
  Centers for Disease Control and Prevention (CDC). (2013). National
  Health and Nutrition Examination Survey: Analytic guidelines,
  2011--2012. U.S. Department of Health and Human
  Services.\url{https://www.cdc.gov/nchs/nhanes}
\item
  Fogoros, R. (2010, October 14). Systolic and Diastolic Blood Pressure.
  Verywell Health; Verywellhealth.
  \url{https://www.verywellhealth.com/systolic-and-diastolic-blood-pressure-1746075}
\item
  Itani, O., Jike, M., Watanabe, N., \& Kaneita, Y. (2017). Short sleep
  duration and health outcomes: A systematic review, meta-analysis, and
  meta-regression. Sleep Medicine, 32, 246--256.
  \url{https://doi.org/10.1016/j.sleep.2016.08.006}
\item
  MedPsych Health. (n.d.). Woman sleeping peacefully in bed
  {[}Photograph{]}. MedPsych
  Health.\href{https://www.medpsychhealth.com/wp-content/uploads/shutterstock_1427337869-1440x810.jpg}{https://www.medpsychhealth.com/
  wp-content/uploads/shutterstock\_1427337869-1440x810.jpg}
\item
  Naseem, A. (2024, May 22). 5 reasons men avoid going to the doctor
  \textbar{} Nuffield Health. Www.nuffieldhealth.com.
  \url{https://www.nuffieldhealth.com/article/5-reasons-men-avoid-going-to-the-doctor}
\end{itemize}

\end{document}
