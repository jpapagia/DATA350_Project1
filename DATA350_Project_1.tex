% Options for packages loaded elsewhere
\PassOptionsToPackage{unicode}{hyperref}
\PassOptionsToPackage{hyphens}{url}
%
\documentclass[
]{article}
\usepackage{amsmath,amssymb}
\usepackage{iftex}
\ifPDFTeX
  \usepackage[T1]{fontenc}
  \usepackage[utf8]{inputenc}
  \usepackage{textcomp} % provide euro and other symbols
\else % if luatex or xetex
  \usepackage{unicode-math} % this also loads fontspec
  \defaultfontfeatures{Scale=MatchLowercase}
  \defaultfontfeatures[\rmfamily]{Ligatures=TeX,Scale=1}
\fi
\usepackage{lmodern}
\ifPDFTeX\else
  % xetex/luatex font selection
\fi
% Use upquote if available, for straight quotes in verbatim environments
\IfFileExists{upquote.sty}{\usepackage{upquote}}{}
\IfFileExists{microtype.sty}{% use microtype if available
  \usepackage[]{microtype}
  \UseMicrotypeSet[protrusion]{basicmath} % disable protrusion for tt fonts
}{}
\makeatletter
\@ifundefined{KOMAClassName}{% if non-KOMA class
  \IfFileExists{parskip.sty}{%
    \usepackage{parskip}
  }{% else
    \setlength{\parindent}{0pt}
    \setlength{\parskip}{6pt plus 2pt minus 1pt}}
}{% if KOMA class
  \KOMAoptions{parskip=half}}
\makeatother
\usepackage{xcolor}
\usepackage[margin=1in]{geometry}
\usepackage{longtable,booktabs,array}
\usepackage{calc} % for calculating minipage widths
% Correct order of tables after \paragraph or \subparagraph
\usepackage{etoolbox}
\makeatletter
\patchcmd\longtable{\par}{\if@noskipsec\mbox{}\fi\par}{}{}
\makeatother
% Allow footnotes in longtable head/foot
\IfFileExists{footnotehyper.sty}{\usepackage{footnotehyper}}{\usepackage{footnote}}
\makesavenoteenv{longtable}
\usepackage{graphicx}
\makeatletter
\newsavebox\pandoc@box
\newcommand*\pandocbounded[1]{% scales image to fit in text height/width
  \sbox\pandoc@box{#1}%
  \Gscale@div\@tempa{\textheight}{\dimexpr\ht\pandoc@box+\dp\pandoc@box\relax}%
  \Gscale@div\@tempb{\linewidth}{\wd\pandoc@box}%
  \ifdim\@tempb\p@<\@tempa\p@\let\@tempa\@tempb\fi% select the smaller of both
  \ifdim\@tempa\p@<\p@\scalebox{\@tempa}{\usebox\pandoc@box}%
  \else\usebox{\pandoc@box}%
  \fi%
}
% Set default figure placement to htbp
\def\fps@figure{htbp}
\makeatother
\setlength{\emergencystretch}{3em} % prevent overfull lines
\providecommand{\tightlist}{%
  \setlength{\itemsep}{0pt}\setlength{\parskip}{0pt}}
\setcounter{secnumdepth}{-\maxdimen} % remove section numbering
\usepackage{bookmark}
\IfFileExists{xurl.sty}{\usepackage{xurl}}{} % add URL line breaks if available
\urlstyle{same}
\hypersetup{
  pdftitle={Sleep Duration, Difficulty, and Health Correlates Among U.S. Adults (NHANES 2009--2011)},
  hidelinks,
  pdfcreator={LaTeX via pandoc}}

\title{Sleep Duration, Difficulty, and Health Correlates Among U.S.
Adults (NHANES 2009--2011)}
\author{Yianni Papagiannopoulos\\
Madhavan Narkeeran\\
Alexandra Julka}
\date{October 15, 2025}

\begin{document}
\maketitle

{
\setcounter{tocdepth}{3}
\tableofcontents
}
\begin{center}\rule{0.5\linewidth}{0.5pt}\end{center}

\subsection{\texorpdfstring{\textbf{Introduction}}{Introduction}}\label{introduction}

Sleep is a crucial factor influencing physical, emotional, and cognitive
health. Adequate sleep supports immunological response, metabolic
balance, and mental resilience, while long-term sleep loss is associated
with obesity, heart disease, anxiety, and depression (Cappuccio et al.,
2010; Itani et al., 2017). Stressors, digital exposure, and changing
work habits have all been recently linked to the widespread sleep
disruptions that affect the length and quality of sleep in modern
societies.

This project uses data from the \texttt{NHANESraw} (2009--2011) survey
dataset, which is conducted by the Centers for Disease Control and
Prevention (CDC) to monitor health and nutrition trends in the United
States (Centers for Disease Control and Prevention {[}CDC{]}, 2013). The
dataset contains 20293 responses (including all of the missing values).
The purpose of this analysis is to investigate and illustrate the
relationship between sleep patterns and demographic characteristics,
lifestyle choices, and overall health, not to establish cause and
effect. The assumption, however, is that BMI (Body-Mass-Index), mental
health, age, gender, and amount of physical activity all have a
significant effect on sleep.

\subsection{\texorpdfstring{\textbf{Background}}{Background}}\label{background}

The Centers for Disease Control and Prevention (CDC) operates the
National Center for Health Statistics (NCHS) program to administer the
National Health and Nutrition Examination Survey (NHANES). The U.S.
population's health and nutritional status are assessed through a
combination of laboratory testing, structured interviews, and physical
examinations (Centers for Disease Control and Prevention {[}CDC{]},
2013). The NHANES incorporates both self-reported and objective medical
data, in contrast to the majority of health surveys which only utilize
questionnaires. This gives a comprehensive picture of how biological and
lifestyle factors interact to affect the overall health of an
individual.

\subsection{\texorpdfstring{\textbf{Data and Cleaning
Strategy}}{Data and Cleaning Strategy}}\label{data-and-cleaning-strategy}

The NHANESraw dataset contained numerous missing values and data
inconsistencies. For the purpose of preserving as many observations as
possible, data cleaning was approached on a question-by-question basis.
Rows with missing values were eliminated only for the variables
pertinent to each analysis, as opposed to eliminating all incomplete
records at once. For numerical variables, extreme outliers were removed
to enhance the quality of the data. For instance, observations reporting
less than 2 hours or more than 15 hours of sleep were not included in
the \texttt{SleepHrsNight} variable because these values were deemed
unrealistic.

To make comparisons easier, some variables were also re-coded or
normalized. For example, in order to enable faceted visualizations by
age range, the Age variable was binned into an \texttt{AgeGroup} column
with categories of 16--24, 25--34, 35--44, 45--54, 55--64, 65--74, and
75+. Furthermore, inconsistent labels for categorical variables were
standardized. The Race1 variable, which had distinct entries for
``Mexican'' and ``Hispanic,'' was one such example. For the purposes of
maintaining consistency, all ``Mexican'' entries were recorded as
``Hispanic,'' since ``Mexican'' is a subgroup of ``Hispanic.''

Overall, these focused cleaning and normalization processes enhanced
dataset comparability and reliability across analyses, ensuring that the
outcomes represented real relationships rather than artifacts.

\subsection{\texorpdfstring{\textbf{Guiding Research
Questions}}{Guiding Research Questions}}\label{guiding-research-questions}

This project aims to explore how demographic, behavioral, and
health-related factors correlate with sleep duration and sleep
difficulty among U.S. adults. The analysis began with ten exemplary
questions, which were subsequently refined to seven primary research
questions. Specifically, this study addresses the following:

\begin{enumerate}
\def\labelenumi{\arabic{enumi}.}
\item
  \textbf{How does reported sleep duration relate to reported difficulty
  sleeping across age groups?}
\item
  \textbf{How does the proportion of reported trouble sleeping vary by
  age group for males versus females?}
\item
  \textbf{What is the observable relationship, if any, between reported
  average sleep duration and average blood pressure (systolic and
  diastolic) in adult participants?}
\item
  \textbf{What association, if any, exists between daily screen time and
  nightly sleep duration?}
\item
  \textbf{How specifically does Reported General Health associate with
  Reported Sleep Trouble?}
\item
  \textbf{What is the relationship between reported physical activity
  and average sleep duration among adults in the United States?}
\item
  \textbf{Among adults aged 21 years or older, what patterns in sleep
  duration are observed across different lifestyle groups (smokers,
  drinkers, both, or neither)?}
\end{enumerate}

Additional exploratory questions and visualizations that were
considered, but not included in the final analysis are presented in the
Appendix.

\subsection{\texorpdfstring{\textbf{Results and
Visualizations}}{Results and Visualizations}}\label{results-and-visualizations}

\textbf{QUESTION~ 1}

Out of the six guided questions, three of them were ones that matched
the initial assumptions for this project. The other four were somewhat
of a surprise. The first question, for instance, explored the
relationship between age groups and difficulty sleeping. The initial
assumption was that age had a significant effect on sleep and that
participants that reported having trouble sleeping slept for a fewer
number of hours on average. The visualization and statistical test
confirmed this assumption.

\begin{figure}
\includegraphics[width=1\linewidth]{DATA350_Project_1_files/figure-latex/fig1_sleep_age-1} \caption{Figure 1. Average Sleep Duration by Age and Reported Difficulty Sleeping}\label{fig:fig1_sleep_age}
\end{figure}

\emph{Mean nightly sleep duration (±95\% confidence intervals) by age
group and self-reported trouble sleeping. Respondents who reported
having trouble sleeping consistently averaged fewer hours of nightly
rest across all age groups, with both groups showing a slight rebound in
older adulthood (NHANES 2009--2011).}

As expected, people with no issues sleep for a significantly longer time
than people with those issues. What's interesting about this dataset in
particular is that older people who do not have any reported issues with
sleep rest for longer than the younger people, whereas it's the other
way around for those with sleep problems. This could be because young
people are healthier than older people, so when they are working through
sleep problems they are unlikely to also be dealing with other physical
issues that may keep them awake, unlike elderly folk. On the other hand,
older people who do not have to deal with sleep problems are likely to
have more time on their hands than younger people to sleep in, so if
they can sleep for long they likely will. As for middle aged people,
they have health problems and little time to deal with, which is why
they sleep the least.~

\textbf{QUESTION 5}

Similarly, question five also demonstrated to be what was expected
initially. Using common knowledge of how mental and physical health
affects sleep, it was deduced that there should be a positive
correlation between reported general health and reported sleep trouble.
The visualization and analysis is shown below.

\begin{figure}
\includegraphics[width=1\linewidth]{DATA350_Project_1_files/figure-latex/fig8_sleep_health-1} \caption{{\small Figure 8. Reported Sleep Trouble by Self-Reported General Health (NHANES 2009–2011)}}\label{fig:fig8_sleep_health}
\end{figure}

\emph{Percent that reported Sleep Trouble for each category of
Self-Reported General Health. Error bars are 95\% confidence intervals
(NHANES 2009--2011).}

It can be seen that among people who have reported worse general health
have a higher percentage of reported sleep trouble. To check if we have
statistically significant evidence to believe this, we perform the
following statistical test.

\textbf{Statistical Test}: Chi-Squared Test for Trend (Cochran--Armitage
trend test)

\textbf{Hypotheses:}

H₀: There is no linear trend in reported sleep trouble percentage across
reported general health categories.

H₁: There is a positive linear trend (reported sleep trouble percentage
increases as reported general health worsens).

Testing at significance level = 0.05.

\textbf{Conclusion:}

The p-value was found to be less than 2.2e-16 which is much less than
the significance level = 0.05. Therefore, there is very strong
sufficient evidence to conclude that the percentage that reported having
trouble sleeping increases as reported general health worsens.

\textbf{QUESTION 2}

Question 2 was slightly more complicated than the first two questions,
however in retrospect the results make sense. There was no real initial
hypothesis. Rather, the goal was to create a visualization and make
observations from that.

\begin{figure}
\includegraphics[width=1\linewidth]{DATA350_Project_1_files/figure-latex/ig3_sleep_gender-1} \caption{{\small Figure 3.2. Reported Trouble Sleeping by Age Group (16+) and Gender}}\label{fig:ig3_sleep_gender}
\end{figure}

\emph{Proportion of reported trouble sleeping across different age
groups for Males and Females. Error bars are 95\% confidence intervals
(NHANES 2009--2011).}

Based off of Figure 3, it seems that Females had a higher proportion of
reported trouble sleeping across all the age groups compared to Males.
To see if we have statistically significant evidence to believe this, we
perform a proportional z test for each Age Group separately using
significance level = 0.05 testing the null hypothesis that females and
males have equal proportions in reporting trouble sleeping against the
alternative that females have higher proportions of reporting trouble
sleeping compared to males. The results of our proportional z tests are
shown in the table below:

\begin{longtable}[]{@{}ll@{}}
\toprule\noalign{}
\endhead
\bottomrule\noalign{}
\endlastfoot
\textbf{Age Group} & \textbf{P-Value} \\
16-24 & 0.000157 \\
25-34 & 0.000104 \\
35-44 & 0.00000141 \\
45-54 & 0.000288 \\
55-64 & 0.00000582 \\
65-74 & 0.0000101 \\
75+ & 0.00297 \\
\end{longtable}

Based on the results, the p-value for all the proportional z tests are
much less than the 0.05 significance level. Therefore, we have
statistically significant evidence to believe that Females have a higher
proportion of reporting sleep trouble compared to Males for each Age
Group.

As indicated above, there was no initial hypothesis about this test. It
was generally unknown whether women or men or neither suffer more from
sleep problems. Even now, with these results, it's still unknown. It is
important to remember that the dependent variable is reported sleep
problems, not hours slept. So a lot of the results are dependent on the
knowledge and perspective of the person reporting rather than simply the
sleeping problems themselves. It's possible that because of societal
standards, the men in the survey may not have felt as comfortable
reporting their potential physical and mental issues. This is supported
by an article~ titled ``5 Reasons Men Avoid Going to the Doctor'' on the
website Nuffield Health, where it's stated that ``65\% of men say they
avoid seeking medical help for as long as possible when they're
unwell.'' (Naseem, 2024). This tendency of men to avoid reporting their
medical issues needs to be addressed because it indicates that just
because women report more of their sleep problems, doesn't mean they
necessarily experience them more than men. When it comes to age, on the
other hand, the results were expected. Middle aged people reported the
most amount of sleep problems. This result correlates with what the
result from the first question indicated: that middle-aged people sleep
less due to lack of time and health that their older and younger
counterparts respectively possess.

\textbf{QUESTION 7}

Question 7 was also complicated, although unlike question 2, it was more
complicated than expected. The initial hypothesis was that people who
engaged with unhealthy habits (such as smoking and drinking) would have
unhealthier sleep tendencies. The density plot and statistical test
below shows otherwise.

\begin{figure}
\includegraphics[width=1\linewidth]{DATA350_Project_1_files/figure-latex/fig9_sleep_lifestyle-1} \caption{Figure 9. Sleep Duration Distribution by Lifestyle (Adults 21+)}\label{fig:fig9_sleep_lifestyle}
\end{figure}

As can be seen, the curve for smoking is less varied than the curve for
drinking or neither. This means that the majority of people who smoke
sleep on average for 7 hours, and fewer people sleep the other hours.
Whereas with people who drink in particular, there are spikes in the
6,7,and 8 hour zones, which means that people who drink engage in sleep
habits that are less similar to other people who drink when compared to
people who smoke, who have more similar habits. Even people who engaged
in neither activity had a more varied sleep schedule on average than
people who smoke. This was surprising because it was expected that there
would be more variability in sleep habits for smokers, drinkers, and
people who engage in both since smoking and drinking are unhealthy.
However, these surprising results bring about an important lesson to
take note of: just because a habit is unhealthy doesn't mean that it
correlates to every other lifestyle choice being unhealthy. In fact, in
this case, it's possible that smokers follow a tighter routine than
non-smokers due to their addiction. That's just speculation, though.~\\

\textbf{QUESTION 6}

The results for the next questions were a bit surprising. For question 6
in particular, it was expected that physical activity positively affects
sleep quantity and quality. However, the box plot below seems to
indicate that there is actually no correlation.~

These results are interesting because many studies indicate that more
exercise results in better sleep. However, there are a few things to
take note of. Average sleep duration is a measure of healthy sleep
habits, but it isn't the only measure of health. After all, there's
still sleep quality to consider, and that can be very different from
sleep quantity. Second, sleep is the dependent variable in this case. It
is possible that sleep has more of an effect on physical activity than
physical activity has on sleep. Sleep is the cause of a lot of different
things, physical prowess being one of them. So a good takeaway from
these results could be that just because one thing doesn't have a
significant effect on something else, doesn't mean the reverse can't be
true.~\\

\textbf{QUESTION 4}

The results to this question was possibly the most surprising out of all
the results. The assumption was that screen usage affects sleep.~ This
is because most devices emit blue light, which diminishes the amount of
melatonin in the body. Melatonin, for reference, helps regulate the
sleeping cycle and aids the body in falling asleep. The visualization
below tells a different story, however.~

\begin{figure}
\includegraphics[width=1\linewidth]{DATA350_Project_1_files/figure-latex/fig7_screen_sleep-1} \caption{Figure 7. Daily Screen Time (Computer + TV) vs. Sleep Duration}\label{fig:fig7_screen_sleep}
\end{figure}

Heat map illustrating the relationship between total daily screen time
(hours spent watching TV and using a computer) and average nightly sleep
duration among participants aged 16 and older. The percentage of
respondents who fall into that screen time--sleep range is shown by each
tile. The majority of people cluster around 7 hours of sleep per night,
according to the fitted regression line (slope ≈ 0.005, R2 ≈ 0.004),
which indicates no discernible relationship between total screen
exposure and sleep duration (NHANES 2009--2011)

As can be seen, the regression for this visualization is mostly flat. So
as screen usage increases, the number of hours people sleep does not
change by too much. This was unexpected, however, it still does make
sense beyond an error or unlikelihood occurring with the analysis. As
indicated in the question 6 analysis, sleep quantity is not the same as
sleep quality. So it's still very much possible that screens affect
sleep quality, but don't affect sleep quantity as much. The results are
still surprising, because one would expect people who use their phones
often, especially those who have a phone addiction, to put off sleep in
order to stay on their phones.~

\textbf{Question 3}

It was expected that blood pressure would have an effect on sleep. There
are two values that measure someone's blood pressure: systolic and
diastolic blood pressure. According to an article by VeryWellHealth on
systolic vs diastolic blood pressure, systolic blood pressure is more
important to doctors because it measures the heart when it's at its
highest pressure point (during the heartbeat). However, for this
project, both systolic and diastolic pressures are analyzed. The results
are shown below.

\begin{figure}
\includegraphics[width=1\linewidth]{DATA350_Project_1_files/figure-latex/fig4_sleep_bp-1} \caption{Figure 4. Sleep Hours vs. Blood Pressure (Adults 16+)}\label{fig:fig4_sleep_bp}
\end{figure}

Scatterplots displaying the relationship between nightly sleep duration
(hours) and average blood pressure (systolic \& diastolic) among adults
aged 16 and older. An ordinary least squares (OLS) regression line with
blue bands representing 95\% confidence intervals is included in each
panel. While there are weak negative correlations between sleep duration
and both diastolic and systolic blood pressure (p \textless{} 0.01), the
data are not clearly linear and are widely scattered. The regression
lines only explain less than 1\% of the variance and summarize a very
slight downward trend, suggesting that blood pressure is not a reliable
indicator of sleep duration.

The expectation was that both measurements of blood pressure would have
a negative effect on sleep (so the higher the blood pressure, the fewer
the hours of sleep). However, as can be seen in the scatterplot above,
it appears that there is a negligible negative correlation between both
measurements and sleep. The p-value is relatively low, which usually
indicates that there is a significant correlation between the two
variables. However, with the r\^{}2 value being so low, this can be
attributed to the dataset being so huge (there are still thousands of
data points even when the missing values are removed). So with all this
being considered, blood pressure has no profound effect on sleep. This
was surprising, however it should be noted that people can control when
they sleep. Most people are aware of the benefits of sleeping for a full
8 hours, and most people strive to sleep for those full 8 hours. Blood
pressure isn't going to stop someone from deciding they need to get to
bed earlier; that's a decision. It would be more interesting to explore
how sleep affects blood pressure, since blood pressure isn't something
people can directly control like the decision of when to go to bed.~\\

\subsection{\texorpdfstring{\textbf{Limitations and Sources of
Error}}{Limitations and Sources of Error}}\label{limitations-and-sources-of-error}

When interpreting the findings of this analysis, it is important to take
into account a number of limitations and possible sources of error.

First, some variables in the \texttt{NHANESraw} dataset used in this
study are dated, as it was gathered between 2009 and 2011. Given the
significant changes in smartphone use and digital habits since then,
this is especially pertinent for metrics such as daily screen time. As a
result, some relationships, including those between screen time and
sleep duration, may not fully reflect patterns observed in 2025. Second,
as stated earlier, self-reported data from questionnaires and interviews
make up a portion of the dataset. Recall bias, social desirability bias,
and rounding errors may affect these kinds of data. For example,
participants may tend to estimate total hours of sleep instead of
reporting precise durations.

Furthermore, although precautions were taken to guarantee data
consistency, such as eliminating missing entries based on
question-specific criteria and filtering implausible values, these
actions might have resulted in selection bias or decreased the
representativeness of particular subgroups. Additionally, re-coding and
grouping variables (e.g., binning age into categories or combining
``Mexican'' into ``Hispanic'') made the data easier to analyze, but they
might have masked more subtle differences within groups. To add on to
this, the NHANES' intricate survey weights were not applied; instead,
the dataset was analyzed as a straightforward random sample. Therefore,
the results are not strictly generalizable to the entire U.S.
population; rather, they describe associations within the sample itself.

Next, cross-sectional data, which offers a moment in time, forms the
basis of the study. Although correlations can be found, such as between
blood pressure and sleep duration, causal relationships cannot be
deduced. Results may also be impacted by unmeasured confounding factors
such as stress, occupation, or long-term medical conditions. Finally,
human judgment in establishing thresholds, exclusions, and groupings
invariably introduces some subjectivity, even in the face of efforts to
maintain objectivity during data cleaning and interpretation.~

All things considered, even though these restrictions limit the accuracy
and generalizability of some findings, the findings nevertheless provide
insightful information about the sleep patterns and their relationships
to various demographic and health characteristics among American adults.

\subsection{\texorpdfstring{\textbf{Conclusion}}{Conclusion}}\label{conclusion}

Despite growing public awareness, many aspects of how sleep interacts
with behavior and self-perceived health remain unclear. For instance,
people may have comparable sleep schedules but greatly differ in their
degrees of well-being and rest. The importance of looking at both hours
slept and self-reported difficulty sleeping is highlighted by the
distinction between sleep quantity and quality. It's also important to
keep in mind that causal relationships don't always go both ways. For
instance, it's heavily implied in an article written for the Cleveland
Clinic that sleep has a significant effect on blood pressure. However,
the results of this project indicate that blood pressure has a limited
effect on the number of hours a person sleeps. The main takeaway of this
project is to never take common sense for granted. It's always better to
test theories, no matter how obvious they may seem at first, because the
results may be different than expected. A good test may include certain
statistical methods, but an even great test is to simply visualize the
theory by plotting a dataset. As the famous saying goes, ``a picture is
worth a thousand words''.

\subsection{\texorpdfstring{\textbf{Reproducibility and Code
Availability}}{Reproducibility and Code Availability}}\label{reproducibility-and-code-availability}

All the source code is available on
\href{https://github.com/jpapagia/DATA350_Project1.git}{GitHub} with the
main Rmd file to reproduce all the results from this report as well as
separate R files to be able to reproduce results for each individual
question explored.\\

This analysis was conducted in R utilizing the following packages:
\texttt{tidyverse}, \texttt{gridExtra}, \texttt{scales}, and
\texttt{broom}.\\

The included \texttt{NHANESraw.csv} dataset must be in the working
directory before execution in order for the results to be reproduced.
All of the figures, tables, and statistical outputs in this report can
be completely recreated with these dependencies and resources in place.

\subsection{\texorpdfstring{\textbf{Appendix}}{Appendix}}\label{appendix}

Although they were created during the analysis's exploratory stage, the
following analysis questions and accompanying visualizations were
eventually omitted from the final report. These figures are included
here to show the variety of exploratory methods used on the NHANES
dataset and to document the larger analytical process. Although they
offered helpful intermediate insights, they either failed to identify
strong patterns, were redundant with other visuals, or were not
sufficiently informative for the research questions.

\textbf{What is the relationship between average sleep duration and the
number of poor mental health days, and how does this relationship differ
across age groups?}\\

\begin{figure}
\includegraphics[width=1\linewidth]{DATA350_Project_1_files/figure-latex/fig2_sleep_mentalhealth-1} \caption{Figure 2. Sleep Duration vs. Days of Poor Mental Health (Past 30 Days) per Age Group}\label{fig:fig2_sleep_mentalhealth}
\end{figure}

\emph{Scatterplot with LOESS smoothing lines showing the association
between average nightly sleep duration and the number of self-reported
days of poor mental health within a 30 day period, separated by age
group. Across all groups, much shorter or longer sleep durations than
the average were generally associated with more days reported with poor
mental health. On average, the lowest number of poor mental health days
in the 30 day period occurred when participants revived nearly 7 hours
of sleep per night. The accompanying table lists Spearman correlation
coefficients (ρ), p-values, and sample sizes for each age group (NHANES
2009--2011).}

\textbf{How does the proportion of reported trouble sleeping vary by age
group for different races?}

\begin{figure}
\includegraphics[width=1\linewidth]{DATA350_Project_1_files/figure-latex/fig3_sleep_race-1} \caption{{\small Figure 3.1. Reported Trouble Sleeping by Age Group (16+) and Race/Ethnicity}}\label{fig:fig3_sleep_race}
\end{figure}

\emph{Bar chart showing the proportion of adults reporting trouble
sleeping by race/ethnicity across seven age groups. Mexican participants
were merged into the Hispanic category. Each bar represents the
estimated percentage within a group, with 95 \% Wald confidence
intervals shown as error bars. Prevalence of reported sleep trouble
tends to increase from young adulthood through middle age and remains
elevated into older adulthood, with White respondents generally
reporting the highest rates across most age categories (NHANES
2009--2011).}

\subsection{\texorpdfstring{\textbf{References}}{References}}\label{references}

\begin{itemize}
\item
  Buysse, D. J. (2014). Sleep health: Can we define it? Does it matter?
  Sleep, 37(1), 9--17. \url{https://doi.org/10.5665/sleep.3298}
\item
  Cappuccio, F. P., D'Elia, L., Strazzullo, P., \& Miller, M. A. (2010).
  Sleep duration and all-cause mortality: A systematic review and
  meta-analysis of prospective studies. Sleep, 33(5), 585--592.
  \url{https://doi.org/10.1093/sleep/33.5.585}
\item
  Clinic, C. (2023, February 13). How a Lack of Sleep Contributes to
  High Blood Pressure. Cleveland Clinic; Cleveland Clinic.
  \url{https://health.clevelandclinic.org/can-lack-of-sleep-cause-high-blood-pressure}
\item
  Centers for Disease Control and Prevention (CDC). (2013). National
  Health and Nutrition Examination Survey: Analytic guidelines,
  2011--2012. U.S. Department of Health and Human
  Services.\url{https://www.cdc.gov/nchs/nhanes}
\item
  Fogoros, R. (2010, October 14). Systolic and Diastolic Blood Pressure.
  Verywell Health; Verywellhealth.
  \url{https://www.verywellhealth.com/systolic-and-diastolic-blood-pressure-1746075}~
\item
  Itani, O., Jike, M., Watanabe, N., \& Kaneita, Y. (2017). Short sleep
  duration and health outcomes: A systematic review, meta-analysis, and
  meta-regression. Sleep Medicine, 32, 246--256.
  \url{https://doi.org/10.1016/j.sleep.2016.08.006}
\item
  MedPsych Health. (n.d.). Woman sleeping peacefully in bed
  {[}Photograph{]}. MedPsych
  Health.\href{https://www.medpsychhealth.com/wp-content/uploads/shutterstock_1427337869-1440x810.jpg}{https://www.medpsychhealth.com/
  wp-content/uploads/shutterstock\_1427337869-1440x810.jpg}
\item
  Naseem, A. (2024, May 22). 5 reasons men avoid going to the doctor
  \textbar{} Nuffield Health. Www.nuffieldhealth.com.
  \url{https://www.nuffieldhealth.com/article/5-reasons-men-avoid-going-to-the-doctor}~
\end{itemize}

\end{document}
